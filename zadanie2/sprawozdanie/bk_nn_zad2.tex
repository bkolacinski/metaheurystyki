\documentclass{article}
\usepackage[utf8]{inputenc}
\usepackage[T1]{fontenc}
\usepackage[polish]{babel}
\usepackage[a4paper, margin=3cm]{geometry}
\usepackage{array}
\usepackage{tocloft}
\usepackage{amsmath}
\usepackage{graphicx}
\usepackage{listings}
\usepackage{xcolor}
\usepackage{hyperref}

\definecolor{codegreen}{rgb}{0,0.6,0}
\definecolor{codegray}{rgb}{0.5,0.5,0.5}
\definecolor{codepurple}{rgb}{0.58,0,0.82}
\definecolor{backcolour}{rgb}{0.95,0.95,0.92}

\lstdefinestyle{mystyle}{
    backgroundcolor=\color{backcolour},   
    commentstyle=\color{codegreen},
    keywordstyle=\color{magenta},
    numberstyle=\tiny\color{codegray},
    stringstyle=\color{codepurple},
    basicstyle=\ttfamily\footnotesize,
    breakatwhitespace=false,         
    breaklines=true,                 
    captionpos=b,                    
    keepspaces=true,                 
    numbers=left,                    
    numbersep=5pt,                  
    showspaces=false,                
    showstringspaces=false,
    showtabs=false,                  
    tabsize=2
}

\lstset{style=mystyle}
\renewcommand{\lstlistingname}{Kod}

\hypersetup{
    colorlinks=true,
    linkcolor=black,
    urlcolor=blue
}

\renewcommand{\cftsecleader}{\cftdotfill{\cftdotsep}}

\begin{document}
    \title{Metaheurystyki --- zadanie 2 \\
    \large Algorytm symulowanego wyżarzania}
    \date{\today}
    \author{
        Bartosz Kołaciński \\
        251554
        \and
        Nikodem Nowak \\
        ??????
    }
    \maketitle

    \vfill
    \begin{center}
        \begin{tabular}{|l|r|}
            \hline
            \textbf{Użyte technologie: } & Python 3.13\\ \hline
            \textbf{Użyte biblioteki: } & \parbox[t]{3cm}{\raggedleft time\\math\\numpy\\random\\collections.abc\\matplotlib.pyplot}\\ \hline
        \end{tabular}
    \end{center}
    \vspace{1cm}

    \newpage

    \tableofcontents

    \newpage
    \section{Opis zasad działania algorytmu}
    \subsection{Opis algorytmu symulowanego wyżarzania}

    \subsection{Opis implementacji rozwiązania}

    Cała logika algorytmu symulowanego wyżarzania została zaimplementowana w klasie\\\texttt{SimulatedAnnealing}.
    \texttt{self.func} to funkcja której maksimum chcemy znaleźć, \texttt{self.domain} to obszar w którym szukamy maksimum, a \texttt{self.dimensions} to liczba wymiarów problemu optymalizacyjnego.

    \begin{lstlisting}[language=Python, caption={Inicjalizacja klasy SimulatedAnnealing}]
    class SimulatedAnnealing:
        def __init__(self,
                 func: Callable,
                 domain: tuple[float, float] | list[tuple[float, float]]):
        self.func = func
        if isinstance(domain, tuple) and len(domain) == 2 and isinstance(
                domain[0], (int, float)):
            self.domain = [domain]
        else:
            self.domain = domain
        self.dimensions = len(self.domain)
    \end{lstlisting}
    \subsection{Opis uruchomienia programu}
    

    \newpage
    \section{Odtworzenie eksperymentów z artykułu}
    \subsection{Wybrana funkcja z rozdziału 3}

    \vspace{1em}
    Z sekcji 3 wybraliśmy funkcję \(f(x)\) z przykładu 1, określoną w przedziale \([-150, 150]\), wyrażoną wzorem:
    \vspace{0.5em}
    \[
    f(x) =
    \begin{cases}
        -2|x+100|+10 & \text{ dla } x\in(-105,-95) \\
        -2.2|x-100|+11 & \text{ dla } x\in(95,105) \\
        0 & \text{ dla } x\notin(-105,-95)\cup(95,105)
    \end{cases}
    \]
    \vspace{-1.75em}
    \begin{figure}[h!]
        \centering
        \includegraphics[width=0.8\textwidth]{figures/func_section_3.png}
        \caption{Wykres funkcji \(f(x)\) w przedziale \([-150, 150]\) z oznaczonym znalezionym maksimum}
        \label{fig:wykres_sekcja_3}
    \end{figure}

    Wybrana funkcja posiada dwa ekstrema lokalne - \(f(-100) = 10\) oraz \(f(100) = 11\). Funkcja poza okolicami tych ekstremów przyjmuje wartość \(0\). Jej maksimum globalne to \(f(100) = 11\).

    Znalezione przez nas maksimum funkcji \(f(x)\) to \(f(100.07) = 10.84\) (rysunek \ref{fig:wykres_sekcja_3}), co różni się od maksimum globalnego o \(\delta=|10.84-11|=0.16\).

    \newpage
    \subsection{Wybrana funkcja z rozdziału 4}

    \vspace{1em}
    Z sekcji 4 wybraliśmy funkcję \(f(x)\) z przykładu 5, określoną w obszarze \([-3,12]\times[4.1,5.8]\), wyrażoną wzorem:
    \vspace{0.5em}
    \[
    f(x,y)=21.5+x\cdot\sin(4\cdot\pi\cdot x)+y\cdot\sin(20\cdot\pi\cdot y)
    \]
    \vspace{-1.75em}
    \begin{figure}[h!]
        \centering
        \includegraphics[width=0.8\textwidth]{figures/func_section_4.png}
        \caption{Wykres funkcji \(f(x,y)\) w obszarze \([-3,12]\times[4.1,5.8]\) z oznaczonym znalezionym maksimum oraz maksimum globalnym}
        \label{fig:wykres_sekcja_4}
    \end{figure}

    Wybrana funkcja posiada bardzo dużo ekstremów lokalnych w określonym obszarze. Jej maksimum globalne to \(f(12,5.7) = 38.85\).

    Znalezione przez nas maksimum funkcji \(f(x,y)\) to \(f(11.12,5.62) = 38.20\) (rysunek \ref{fig:wykres_sekcja_4}), co różni się od maksimum globalnego o \(\delta=|38.20-38.85|=0.65\).




\end{document}