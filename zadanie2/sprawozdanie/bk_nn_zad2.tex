\documentclass{article}
\usepackage[utf8]{inputenc}
\usepackage[T1]{fontenc}
\usepackage[polish]{babel}
\usepackage[a4paper, margin=3cm]{geometry}
\usepackage{array}
\usepackage{tocloft}
\usepackage{amsmath}
\usepackage{graphicx}
\usepackage{hyperref}

\hypersetup{
    colorlinks=true,
    linkcolor=black,
    urlcolor=blue
}

\renewcommand{\cftsecleader}{\cftdotfill{\cftdotsep}}

\begin{document}
    \title{Metaheurystyki --- zadanie 2 \\
    \large Algorytm symulowanego wyżarzania}
    \date{\today}
    \author{
        Bartosz Kołaciński \\
        251554
        \and
        Nikodem Nowak \\
        ??????
    }
    \maketitle

    \vfill
    \begin{center}
        \begin{tabular}{|l|r|}
            \hline
            \textbf{Użyte technologie: } & Python 3.13\\ \hline
            \textbf{Użyte biblioteki: } & \parbox[t]{3cm}{\raggedleft math\\numpy\\random\\collections.abc\\matplotlib.pyplot}\\ \hline
        \end{tabular}
    \end{center}
    \vspace{1cm}

    \newpage

    \tableofcontents

    \newpage
    \section{Odtworzenie eksperymentów z artykułu}
    \subsection{Wybrana funkcja z rozdziału 3}

    \vspace{1em}
    Z sekcji 3 wybraliśmy funkcję \(f(x)\) z przykładu 1, określoną w przedziale \([-150, 150]\), wyrażoną wzorem:
    \[
    f(x) =
    \begin{cases}
        -2|x+100|+10 \text{ dla } x\in(-105,-95) \\
        -2.2|x-100|+11 \text{ dla } x\in(95,105) \\
        0 \text{ dla } x\notin(-105,-95)\cup(95,105)
    \end{cases}
    \]
    \vspace{-1.75em}
    \begin{figure}[h!]
        \centering
        \includegraphics[width=0.8\textwidth]{figures/func_section_3.png}
        \caption{Wykres funkcji \(f(x)\) w przedziale \([-150, 150]\)}
        \label{fig:wykres_fx}
    \end{figure}




\end{document}