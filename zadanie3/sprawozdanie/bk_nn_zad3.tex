\documentclass{article}
\usepackage[utf8]{inputenc}
\usepackage[T1]{fontenc}
\usepackage[polish]{babel}
\usepackage[a4paper, margin=3cm]{geometry}
\usepackage{array}
\usepackage{tocloft}
\usepackage{amsmath}
\usepackage{graphicx}
\usepackage{listings}
\usepackage{xcolor}
\usepackage{booktabs}
\usepackage{hyperref}

\definecolor{codegreen}{rgb}{0,0.6,0}
\definecolor{codegray}{rgb}{0.5,0.5,0.5}
\definecolor{codepurple}{rgb}{0.58,0,0.82}
\definecolor{backcolour}{rgb}{0.95,0.95,0.92}

\lstdefinestyle{mystyle}{
    backgroundcolor=\color{backcolour},   
    commentstyle=\color{codegreen},
    keywordstyle=\color{magenta},
    numberstyle=\tiny\color{codegray},
    stringstyle=\color{codepurple},
    basicstyle=\ttfamily\footnotesize,
    breakatwhitespace=false,         
    breaklines=true,                 
    captionpos=b,                    
    keepspaces=true,                 
    numbers=left,                    
    numbersep=5pt,                  
    showspaces=false,                
    showstringspaces=false,
    showtabs=false,                  
    tabsize=2
}

\lstset{style=mystyle}
\renewcommand{\lstlistingname}{Kod}

\hypersetup{
    colorlinks=true,
    linkcolor=black,
    urlcolor=blue
}

\renewcommand{\cftsecleader}{\cftdotfill{\cftdotsep}}

\begin{document}
    \title{Metaheurystyki --- zadanie 3 \\
    \large Algorytm genetyczny \\
    \small GRUPA 3 --- piątek 10:15}
    \date{\today}
    \author{
        Bartosz Kołaciński \\
        251554
        \and
        Nikodem Nowak \\
        251598
    }
    \maketitle

    \vfill
    \begin{center}
        \begin{tabular}{lr}
            \toprule
            \textbf{Użyte technologie} & Python 3.13 \\
            \midrule
            \textbf{Użyte biblioteki} & \begin{tabular}[t]{@{}r@{}}
                przykladowa \\
                tresc
            \end{tabular} \\
            \bottomrule
        \end{tabular}
    \end{center}
    \vspace{1cm}

    \newpage

    \tableofcontents

    \newpage

    \section{Opis zasad działania algorytmu}
    \subsection{Opis algorytmu genetycznego}

    TODO

    \subsection{Założenia podstawowe}

    TODO

    \subsection{Opis implementacji rozwiązania}

    TODO \texttt{self.func}

    \begin{lstlisting}[language=Python, caption={TODO}]
    pass
    \end{lstlisting}

    \subsection{Instrukcja uruchomienia programu}

    TODO \texttt{python run.py}

    \newpage

    \section{Eksperymenty i wyniki}

    Dla każdego zestawu parametrów algorytm został uruchomiony 5 razy w celu zredukowania losowości wyników. Przyjęte oznacznia dla poszczególnych parametrów to:
    \begin{itemize}
        \item \(oznacznie\) -- prawdopodobieństwo krzyżowania,
        \item \(oznacznie\) -- prawdopodobieństwo mutacji,
        \item \(oznacznie\) -- wielkość populacji,
        \item \(oznacznie\) -- liczba iteracji.
    \end{itemize}

    \noindent
    TODO

    % \begin{figure}[h!]
    %     \centering
    %     \includegraphics[width=0.8\textwidth]{figures/<nazwa_pliku>.png}
    %     \caption{TODO}
    %     \label{fig:<nazwa_etykiety>}
    % \end{figure}

    \newpage

    \section{Analiza wyników i wnioski}

    TODO

\end{document}